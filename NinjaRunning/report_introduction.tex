\documentclass{article}

\usepackage[utf8]{inputenc}
\usepackage{listings}
\usepackage{xcolor}
\usepackage{graphicx}

\title{Harbin Institute of Technology\\
Interactive Design and Development of Digital Media\\
Course Report}
\author{Song Jian}
\date{July 2018}

\begin{document}

\maketitle
\tableofcontents
\begin{table*}[b]
\begin{centering}
\begin{tabular}{|p{0.2\textwidth}|p{0.3\textwidth}|p{0.35\textwidth}|}
\hline
Student No.&Name&email\\
\hline
1150310303&Song Jian&2511146542@qq.com\\
\hline
1150310704&Gong Tianying&540655966@qq.com\\
\hline
\end{tabular}
\end{centering}
\end{table*}
\clearpage



\section{Introduction}
The primitive archetype of the game was a simplified 2-D parkour game. In the game, the player acted as a somehow clumsy dinosaur and aimed at the goal to run as long as he can while avoiding incoming obstacles and perils. Plain as the original game was, we were enlightened to decorate the unornamented game with some more enjoyable machinations. We created a battle system and a perk and ability system to complex the game and set a reachable goal to incite players to try harder until the goal is accomplished. The player, in this modified game version, choose a perk and act as a ninja or shooter, dealing with meandering ghosts whose collisions are hostile and a dragon who shoots mortar bullets at you. Surviving and killing monsters grant you credits and there is an ending of the game, though not all players might be able to see it and begin to doubt its existence.



\section{Game Description}
The player controls a character to shoot enemies and avoid damage, in the meantime running and surviving before your health points reach zero. Both implementation and gameplay details are shown below. One common knowledge, the game is completely designed by us and powered by Unity3D engine and we used the programming language C\#.
\subsection{Character}
The two characters, named Ninja and Gunslinger inherit from the abstract class Character respectively, are available currently in the MainMenu scene.
\subsubsection{Ninja}
Ninja, the primary character as indicated by the game title, is a flexible and powerful pawn. Despite the low damage of his darts, Ninja experts in defending and can even parry bullets.
\begin{description}
\item[Triple Shot] Throw three darts continuously with a slight interval, and naturally implemented with Coroutine
\item[Split Shot] Throw three darts in three different directions, providing good protection from close targets
\item[Twice Jump] Ninja can jump twice before landing, and the second jump is a flip
\item[Blink] Sprint through enemies, dealing damage and avoiding damage of physical collisions with ghosts
\item[Defend] Rebounds bullets from enemies
\end{description}

\subsubsection{Gunslinger}
Gunslinger can shoot .500 Magnum bullet with distance penalty, high damage per one shot, but low fire rate.
\begin{description}
\item[Shoot] Shot .500 Magnum bullet, with the max damage of 120 and min damage of 50. The numeric value of the damage is linearly waned with the incrementation of the distance the bullet spreads
\item[Shield] Can parry one shot when the shield is ready. Cooldown time is 15s
\item[Bullet Eye] Slow down time for most actions for 1.4s, during which Gunslinger can shoot for at most three times
\end{description}


\subsection{Enemy}
There are two kinds of enemies derived from the abstract class Enemy, called Ghost and Patriarch.
\subsubsection{Ghost}
Ghost is a small enemy spawned at a random height and floats up and down. Albeit not heavily armored, their collisions can cause you 30 damage and your total vitality is 100. Note that there are not medic packages betwixt the way you travel, so shooting at the ghosts is an excellent option.

\subsubsection{Patriarch}
Patriarch, the unique boss in this game, can shoot 3 bullets at one time. Well the good news is one patriarch killed grants you 1500 scores, while the bad news that a patriarch slashed means a following one coming soon, so get prepared for an everlasting battle.



\section{Technical Skills}
Here are some basic and high-level skills used in the game.
\subsection{Unity Engine}
\begin{description}
\item[Physics] As required by the original game, we use Rigidbody2D and Collider2D to generate physics effects, including falling from above and collision detection.
\item[Coroutine] Coroutine in Unity Engine is highly useful and efficient to generate latency of a fixed period of time. Examples of Coroutine can be found almost everywhere, such as throwing three darts with a little interval, spawning enemies every several seconds, and defending from bullets for 1.4 seconds.
\item[UGUI] All GUI objects are made with UGUI, including Text, Button, Canvas, CanvasGroup, Image and so on. To perform universal but not perfect UI on machines with varying resolution ratings, we use RectangleTransform.anchor to control the horizontal and vertical ratio of one object according to its parent.
\item[Sound] The collocation of AudioSource and AudioClip enables the game to radio waves when one shot or a hit happens. In addition, the background music is ``Stronger''.
\item[Sprite] Sprites are the significant part of one 2D game. We use some external tools (like Photoshop) to draw and cut the pictures required.
\item[Prefab] We use prefabs to instantiate game objects dynamically with code instead of doing such by dragging game objects from asset folders to scene view.
\item[Other important methods]
Enumerate other methods used frequently.
\begin{itemize}
    \item Game\-Object Game\-Object.Find(string)
    \item T Get\-Component \textless T\textgreater()
    \item Game\-Object Instantiate(Game\-Object, Vector3, Quaternion)
    \item Coroutine Object.Start\-Coroutine(method)\\
    Note that we use method and parameters instead of nameof(method) for performance issues
\end{itemize}

\end{description}

\subsection{Language C\#}
Based on .NET 4.7.1, we use C\# 6.0 to accomplish the workloads of scripts.
\begin{description}
\item[inheritance] There are three abstract classes, Bullet, Enemy and Character for other child classes to derive from, with most of their methods virtual, including On\-Collision\-Enter2D, On\-Trigger\-Enter2D, Awake, Start, Move\-Forward, Update at least.
\item[encapsulation] Some of the member variables are defined as private, while others are public. Though there are six kinds of accessability in C\#, we only used public, protected and private (ample in most occasions).
\item[polymorphism] Virtual methods of abstract classes are widely override in derived classes.
\item[yield] Keyword $yield$ are generally used to create coroutines. In most occasions, we used clause
yield return new Wait\-For\-Seconds(1.4f);
to yield an interval of 1.4 seconds.
\item[lambda expression] C\# provides three kinds of lambda, while only two of them are available in C\# 6.0, normal lambda and member lambda except getter and setter lambda. Clause
public int Distance\-Score =\textgreater (int)(survival\-Distance * distance\-Score\-Weight * distance\-Score\-Factor);
is a good example of member lambda, used to define a public getter property.
\item[delegate and event] Delegate and event are useful attributes and frequently used in this game. We used Enemy\-Die\-Event to tell some objects one enemy just died, and the GUI class to refresh goals. Other events, such as Character\-Survive\-Event, Bullet\-Eye\-Cool\-Down\-Decrease\-Event, also have their unique functions.
\item[exception] We only used Unity\-Engine.Unity\-Exception to raise an exception to avoid some compile error where an unexpected Enumeration object was inputted.
\item[is and as] To judge whether an object is an instance of a class dynamically, keyword $is$ is suitable, after which we use $as$ to explicitly cast the object to the specific class.
\end{description}

\subsection{Other highlights}
\begin{description}
\item[Health bar] Instead of a static Text on the Canvas, we designed a health point bar, the length of which is automatically changed according to the health point of the player. In addition, the color of it modifies from green to yellow, and finally turns red when reaching a really low ratio, which is, technically speaking, implemented with Color.Lerp.
\item[Cool down] We show cool down in a friendly way to player: a grey image shrinking slowly with a big number on it indicating the exact cool down time. When the skill is ready, the 0 number is removed.
\item[Rich Text] We use Rich Text to make the description of characters more colourful for players.
\end{description}



\section{Learning and Thinking}
Through this assignment, we have learned a lot about game development and become more proficient in C\#. I also developed a strong interest in this course. After finishing this course, we should continue to learn more advanced knowledge. Finally, this course at the meantime cultivated our sense of teamwork, during which we made new friends.\\
And we sincerely appreciate the assistance from our teacher.



\section{Future Work}
There are still some potential improvements in our assignment.
\begin{description}
\item[Achievement] Achievements, however, push the players to play the game again and again in ways beyond our imagination. Nevertheless, the event system in our game is designed partially to support expansions of achievements and statistics.
\item[Animation] We failed to generate an animation for characters, such as running, blinking, shooting and so on. The experiment only showed to yield an animation with given key frames by scripting controllers, rather than to create the key frames.
\item[Network] Most games, including ``standalone'' ones, send data to the server and that is one goal we are still working at.
\item[Particle System] A fantastic game can never be without particle system. Still, more work can be done here.
\end{description}
\clearpage



\section{Mission Distribution}
Total contribution:\\
Song 60\%\\
Gong 40\%\\
During the development, we used the Unity collaborate to publish our manipulation respectively and merged our effects together. Here is the website of this project (Ninja\-Running):\\
https://developer.cloud.unity3d.com/col\-lab/orgs/mutation\-delegate\\
/projects/ninja\-running2/ commits/\\
You can see our collaborate history here.
\begin{figure}[tb]
    \centering
    \includegraphics[width=\textwidth]{CollabHistory1.jpg}
    \caption{Collaborate History 1}
\end{figure}
\begin{figure}[tb]
    \centering
    \includegraphics[width=\textwidth]{CollabHistory2.jpg}
    \caption{Collaborate History 2}
\end{figure}
\clearpage



\section{Screen Shots}
\begin{figure}[tb]
    \centering
    \includegraphics[width=0.8\textwidth]{1.jpg}
\end{figure}
\begin{figure}[tb]
    \centering
    \includegraphics[width=0.8\textwidth]{2.jpg}
\end{figure}
\begin{figure}[tb]
    \centering
    \includegraphics[width=0.8\textwidth]{3.jpg}
\end{figure}
\begin{figure}[tb]
    \centering
    \includegraphics[width=0.8\textwidth]{4.jpg}
\end{figure}
\begin{figure}[tb]
    \centering
    \includegraphics[width=0.8\textwidth]{5.jpg}
\end{figure}






















\end{document}
